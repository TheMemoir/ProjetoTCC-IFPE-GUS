%Pacotes
\usepackage[english,brazil]{babel}%Idioma
\usepackage[utf8]{inputenc}%Inserir acentos 
\usepackage[T1]{fontenc}%Tipo de fonte usada na compilação
\usepackage{hyphenat}%Habilita/desabilita hifenização
%Ativar se necessário
%\setlength{\parskip}{1\baselineskip}%Espaçamento de 1pt entre os parágrafos
\usepackage{setspace}%Mudar espaçamento entre linhas
\usepackage[a4paper,%
            left=3.0cm,%
            right=2.0cm,%
            top=3.0cm,%
            bottom=2.0cm]{geometry}
\usepackage{graphicx}%inserir imagens
\usepackage[x11names]{xcolor}%Texto colorido
\usepackage{pdfpages}%Inserir páginas PDF
\usepackage{multicol}%Criar multicolunas sem \tabular
\usepackage{xurl}%Para disponibilizar o endereço do site
\usepackage[breaklinks,hidelinks,colorlinks=true,allcolors=blue]{hyperref}%Configurar hiperlinks
\usepackage{tabularx}%Tabelas mais fáceis
\usepackage{booktabs}
\usepackage{textcmds}%Digitar aspas com \qq e \q
\usepackage{lipsum}%Gerar Lipsum
\usepackage{indentfirst}%Indentar 1º§ de cada seção
\usepackage{microtype}%Melhorar a justificação do texto


%Referências e Citações
\usepackage[alf,
            abnt-emphasize=bf,%
            abnt-etal-list=3,%
            abnt-etal-text=emph,%
            abnt-missing-year=sd,%
            abnt-repeated-author-omit=yes,%
            abnt-repeated-title-omit=yes,%
            abnt-thesis-year=final,%
            abnt-doi=link,%
            ]{abntex2cite}

%Comandos e Ambientes

%Resumo
\renewenvironment{abstract}
  {\small\quotation
  {\bfseries\noindent{\normalsize\abstractname}\par\nobreak\smallskip}}
  {\endquotation}

%Citação +3 linhas
\usepackage{changepage}
\newenvironment{citar}
{\begin{adjustwidth}{4cm}{0cm}\SingleSpacing\footnotesize\par}
{\end{adjustwidth}}

%Texto da folha de rosto (tipo de trabalho)
\newenvironment{textofolharosto}
{\SingleSpacing\small\list{}{\rightmargin=0cm \leftmargin=7cm}%
	\item\relax}%
{\endlist}

%Definições dos capítulos
\chapterstyle{section}%Título somente com o número e o nome, sem "Capítulo X" da classe
\pagestyle{simple}
\aliaspagestyle{part}{empty}
\setsecnumdepth{subsubsection}

%Caixa de texto para uso com códigos
\usepackage{tcolorbox}
\tcbuselibrary{most,listingsutf8}% 
\newtcblisting[auto counter,number within=chapter]{codex}[2][]{%
	arc=0mm,
	listing only,
	colback=black!5!white,
	colframe=black!75!white,
	fonttitle=\bfseries,
	title=Exemplo \thetcbcounter: #1 #2,
	listing options={style=tcblatex,numbersep=1mm,numbers=left,numberstyle=\tiny\color{black}}
}
